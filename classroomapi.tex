\documentclass[11pt,]{article}
\usepackage{graphicx}%from
\usepackage{cite}
\begin{document}
\begin{titlepage}
 \begin{figure}[h]
  \centerline{\small MAKERERE 
  \includegraphics[width=0.1\textwidth]  {muk_log} UNIVERSITY}
\end{figure}
\centerline{COLLEGE OF COMPUTING AND INFORMATION SCIENCES\\}
\paragraph*{•}
\centerline{DEPARTMENT OF COMPUTER SCIENCE\\}
\paragraph*{•}
\centerline{COURSEWORK: RESEARCH METHODOLOGY(BIT 2207)\\}
\paragraph*{•}
\centerline{LECTURER: \textbf{MR.Ernest Mwebaze}}
\paragraph*{•}
\centerline{\begin{tabular}{|c|c|c|c|}
\hline
\textbf{No.}& \textbf{Student Name} & \textbf{RegNo} & \textbf{Student Number} \\ \hline
\textit{1}&\textbf{AGABA DAVIS} & \textit{16/U/2812/PS}& \textit{216009915} \\ \hline
\end{tabular}}
\paragraph*{•}
\centerline{\textbf{GOOGLE PRODUCT: Classroom API}}
\paragraph*{•}
\tableofcontents
\end{titlepage}
\section{Introduction}

Google Classroom\cite{s1} was the marvelous piece of education technology that allowed for more organized and focused classrooms by alleviating clutter. Now, the company has introduced the Google Classroom application program interface.
Google Classroom \cite{s2} has been revered as one of the most popular ways for administrators, developers and educators to organize themselves and be on the same page at all times.
The Google Classroom application program interface(API) was first announced in June during ISTE 2015 in Philadelphia\cite{s3}. A developer preview of the API tool-set ended in July, and the API was made available to the general public Aug. 5.


\section{Platform Overview}
The Google Classroom API\cite{s4} makes it easier for developers and administrators to use Classroom's resources. With these tools, they can quickly manage and the provision  ofGoogle Classroom course at the school or district level.
Now, anyone can begin syncing their Classroom roster and assignments with third-party platforms, including student information systems (SIS) and learning management systems (LMS).
"The end of the preview also means that all Google Apps for Education users can authorize third-party applications to access their Classroom data, unless their admin decides to restrict access in the Admin Console at the organisation unit level," wrote Google Classroom product manager Zach Yeskel \cite{s5}.
\section{Functionality}
For the functionality, access to the API helps teachers seamlessly integrate their apps with Google Classroom. The API is also now supported in Apps Script\cite{s6}, which allows users to create add-ons and modification for Google Docs, Sheets or Forms.
In June,Pear Deck CEO Riley Eynon-Lynch demonstrated how his company was using the API to streamline student invites to Google Classroom  which had been a stumbling block for teachers everywhere\cite{s5}.
Integrating third party apps is like a shortcut to learning. Applications like Google Sheets can be used to integrate classroom data into Google Classroom, making it easier for data to be shared among educators, administrators, and students.




\bibliographystyle{IEEEtran}
\bibliography{DB}
\end{document}